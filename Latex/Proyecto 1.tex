\documentclass[10pt]{article}
\usepackage{bookmark}
% Formato extenso: report

% Formato corto: article

% Esto es para que el LaTeX sepa que el texto está en español:
\usepackage[spanish]{babel}

\usepackage{amsmath, amsthm, amsfonts,amssymb}

% Bórrame si quieres:
\usepackage{multicol}

% Referencias
\usepackage{hyperref}

% Paquete para escribir código
\usepackage{listings}
\lstset{basicstyle=\footnotesize\ttfamily,breaklines=true}
\usepackage{alltt}

\usepackage{tikz}
% Paquete para incluir imágenes
\usepackage{graphicx}

% Paquete para incluir varias imágenes en una
\usepackage{subfig}

% para poder fijar las imágenes ([H])
\usepackage{float}

% para agregar opciones al enumerate
\usepackage{enumerate}


% Teoremas
\newtheorem{thm}{Teorema}[section]
\newtheorem{cor}[thm]{Corolario}
\newtheorem{lem}[thm]{Lema}
\newtheorem{prop}[thm]{Proposición}
\theoremstyle{definition}
\newtheorem{defn}[thm]{Definición}
\theoremstyle{remark}
\newtheorem{rem}[thm]{Observación}
\theoremstyle{definition}
\newtheorem{prob}{Problema}
\numberwithin{equation}{prob}

% Calculus symbols
\newcommand{\pd}[2]{\frac{\partial #1}{ \partial #2}}   % First partial derivative command
\newcommand{\td}[2]{\frac{\mathrm{d} #1}{ \mathrm{d} #2}}
\newcommand{\pdd}[2]{\frac{\partial^2 #1}{ \partial #2 ^2}}   % Second partial derivative command
\newcommand{\pddc}[3]{\frac{\partial^2 #1}{ \partial #2 \partial #3}}   % Second partial derivative command

% Continuum mechanics & FEM symbols
\def\sca   #1{\mbox{\rm{#1}}{}}
\def\mat   #1{\mbox{\boldmath $\mathsf #1$}}
\def\vec   #1{\mbox{\boldmath $#1$}{}}
\def\ten   #1{\mbox{\boldmath $#1$}{}}
\def\ltr   #1{\mbox{\sf{#1}}}
\def\bltr  #1{\mbox{\sffamily{\bfseries{{#1}}}}}

% math operators and symbols
\DeclareMathOperator{\dive}{div}
\DeclareMathOperator{\trace}{trace}
\DeclareMathOperator{\tr}{tr}
\DeclareMathOperator{\symm}{symm}
\DeclareMathOperator{\sk}{skew}
\DeclareMathOperator{\grad}{grad}
\DeclareMathOperator{\Grad}{Grad}
\DeclareMathOperator{\curl}{curl}
\DeclareMathOperator{\Curl}{Curl}
\def\R{\mbox{\(\mathbb{R}\)}}
\def\dx{\mbox{\(\,\mathrm{d}x\)}}


\usepackage{geometry}
\geometry{left=2.5cm, right=2.5cm, top=2cm, bottom=3cm}

\usepackage{makeidx}
\makeindex


\begin{document}

\begin{sloppypar}

\end{sloppypar}

\begin{titlepage}
	%%%%% NO MODIFICAR
	\begin{figure}
		\begin{minipage}{4cm}
			\includegraphics[width=0.9\textwidth]{./figures/logo}
		\end{minipage}
		\begin{minipage}{11cm}
			\vspace{4mm}
			{\sc UNIVERSIDAD DE VALPARAÍSO}\\
			Escuela de Ingeniería Civil Biomédica\\
			{\bf CBM524 - Robótica Médica}\\
			\vspace{0mm}
			\hrulefill
		\end{minipage}
	\end{figure}
	\phantom{""}\vspace{60mm}


	%%%%% MODIFICAR
	\begin{center}
		\Huge{\textbf{Proyecto 1}}\vspace{95mm}\\
		\raggedleft \Large{Jorge Gonzalo Alejandro Alcaíno Brevis}\\
		\raggedleft \Large{Maximiliano Antonio Gaete Pizarro}\\
		\raggedleft \Large{Natalia Muñoz}\\
	\end{center}


\end{titlepage}

\section{Considere un robot articulado con 3 grados de libertad}

\begin{center}
\begin{tikzpicture}

	% Draw the first link (L1) behind the shapes
	\draw[thick,blue, line width=4pt] (0,0.5) -- (0,3.5);
	\node at (-0.5,2) {$L1$};
	
	% Draw the second link (L2) behind the shapes
	\draw[thick,blue, line width=4pt] (0,3.8) -- (4,4);
	\node at (2,4.5) {$L2$};
	
	% Draw the third link (L3) behind the shapes
	\draw[thick,blue, line width=4pt] (4.5,4) -- (7,4.3);
	\node at (5.8,4.7) {$L3$};
	
	% Draw the base (two touching inverted triangles for q1, stylized as in the image)
	\draw[fill=blue!60, thick] (-1,-1) -- (1,-1) -- (0,0.2) -- cycle;  % Upper triangle
	\draw[fill=blue!60, thick] (-1,1) -- (1,1) -- (0,-0.2) -- cycle;   % Lower triangle
	\node at (0,-1.5) {$q1$};
	
	% Draw the first joint (circle for q2)
	\draw[fill=blue!60] (0,3.5) circle (0.5);
	\node at (0,4.5) {$q2$};
	
	% Draw the second joint (circle for q3)
	\draw[fill=blue!60] (4,4) circle (0.5);
	\node at (4,5) {$q3$};
	
	% Draw the end-effector (open shape)
	\draw[thick,blue, line width=4pt] (7,4.3) -- (7.5,5) -- (8,4.3) -- cycle;
	
\end{tikzpicture}
\end{center}

\subsection{Parámetros de Denavit-Hartenberg}

Dado el brazo robótico con tres eslabones, los parámetros D-H son:

\[
\begin{array}{c|c|c|c|c}
\text{Eslabón } i & a_i & \alpha_i & d_i & \theta_i \\
\hline
1 & 0 & 90^\circ & L_1 & \theta_1 \\
2 & L_2 & 0^\circ & 0 & \theta_2 \\
3 & L_3 & 0^\circ & 0 & \theta_3 \\
\end{array}
\]

\subsection{Matrices de transformación homogénea \texorpdfstring{$A_i$}{Ai} para cada eslabón}

La matriz de transformación homogénea estándar entre el eslabón \( i-1 \) y el eslabón \( i \) es:

\[
A_i = \begin{bmatrix}
\cos\theta_i & -\sin\theta_i \cos\alpha_i & \sin\theta_i \sin\alpha_i & a_i \cos\theta_i \\
\sin\theta_i & \cos\theta_i \cos\alpha_i & -\cos\theta_i \sin\alpha_i & a_i \sin\theta_i \\
0 & \sin\alpha_i & \cos\alpha_i & d_i \\
0 & 0 & 0 & 1 \\
\end{bmatrix}
\]

\subsubsection{\texorpdfstring{Matriz $A_1$:}{Matriz A1:}}

Dado que \( \alpha_1 = +90^\circ \):

\[
\cos\alpha_1 = 0, \quad \sin\alpha_1 = 1
\]

Entonces:

\[
A_1 = \begin{bmatrix}
\cos\theta_1 & 0 & \sin\theta_1 & 0 \\
\sin\theta_1 & 0 & -\cos\theta_1 & 0 \\
0 & 1 & 0 & L_1 \\
0 & 0 & 0 & 1 \\
\end{bmatrix}
\]

\subsubsection{\texorpdfstring{Matriz $A_2$:}{Matriz A2:}}

Con \( \alpha_2 = 0^\circ \):

\[
\cos\alpha_2 = 1, \quad \sin\alpha_2 = 0
\]

Entonces:

\[
A_2 = \begin{bmatrix}
\cos\theta_2 & -\sin\theta_2 & 0 & L_2 \cos\theta_2 \\
\sin\theta_2 & \cos\theta_2 & 0 & L_2 \sin\theta_2 \\
0 & 0 & 1 & 0 \\
0 & 0 & 0 & 1 \\
\end{bmatrix}
\]

\subsubsection{\texorpdfstring{Matriz \(A_3\):}{Matriz A3:}}

Con \( \alpha_3 = 0^\circ \):

\[
\cos\alpha_3 = 1, \quad \sin\alpha_3 = 0
\]

Entonces:

\[
A_3 = \begin{bmatrix}
\cos\theta_3 & -\sin\theta_3 & 0 & L_3 \cos\theta_3 \\
\sin\theta_3 & \cos\theta_3 & 0 & L_3 \sin\theta_3 \\
0 & 0 & 1 & 0 \\
0 & 0 & 0 & 1 \\
\end{bmatrix}
\]

\subsection{Cálculo de las posiciones de los eslabones}

\subsubsection{Posición del origen del sistema base (Eslabón 0):}

El sistema base está en el origen, por lo que:

\[
(x_0, y_0, z_0) = (0, 0, 0)
\]

\subsubsection{Posición del Eslabón 1:}

La matriz \( A_1 \) ya está calculada. La posición del eslabón 1 es el vector de traslación en \( A_1 \):

\[
\begin{bmatrix}
x_1 \\
y_1 \\
z_1 \\
1 \\
\end{bmatrix} = A_1 \times \begin{bmatrix}
0 \\
0 \\
0 \\
1 \\
\end{bmatrix} = \begin{bmatrix}
0 \\
0 \\
L_1 \\
1 \\
\end{bmatrix}
\]

Por lo tanto:

\[
(x_1, y_1, z_1) = (0, 0, L_1)
\]

\subsubsection{Posición del Eslabón 2:}

Primero multiplicamos \( A_1 \) y \( A_2 \):

\[
T_0^2 = A_1 \times A_2
\]

Calculando \( T_0^2 \), obtenemos:

\[
T_0^2 = \begin{bmatrix}
\cos\theta_1 \cos\theta_2 & -\cos\theta_1 \sin\theta_2 & \sin\theta_1 & \cos\theta_1 L_2 \cos\theta_2 \\
\sin\theta_1 \cos\theta_2 & -\sin\theta_1 \sin\theta_2 & -\cos\theta_1 & \sin\theta_1 L_2 \cos\theta_2 \\
\sin\theta_2 & \cos\theta_2 & 0 & L_1 + L_2 \sin\theta_2 \\
0 & 0 & 0 & 1 \\
\end{bmatrix}
\]

La posición \( (x_2, y_2, z_2) \) es el vector de traslación en \( T_0^2 \):


\[
\begin{aligned}
x_2 &= \cos\theta_1 \cdot L_2 \cos\theta_2 \\
y_2 &= \sin\theta_1 \cdot L_2 \cos\theta_2 \\
z_2 &= L_1 + L_2 \sin\theta_2 \\
\end{aligned}
\]


\subsubsection{Posición del Eslabón 3:}

Multiplicamos \( T_0^2 \) y \( A_3 \):

\[
T_0^3 = T_0^2 \times A_3
\]

Después de la multiplicación, obtenemos:

\[
T_0^3 = \begin{bmatrix}
\cos\theta_1 \cos(\theta_2 + \theta_3) & -\cos\theta_1 \sin(\theta_2 + \theta_3) & \sin\theta_1 & x_3 \\
\sin\theta_1 \cos(\theta_2 + \theta_3) & -\sin\theta_1 \sin(\theta_2 + \theta_3) & -\cos\theta_1 & y_3 \\
\sin(\theta_2 + \theta_3) & \cos(\theta_2 + \theta_3) & 0 & z_3 \\
0 & 0 & 0 & 1 \\
\end{bmatrix}
\]

La posición \( (x_3, y_3, z_3) \) es:


\[
\begin{aligned}
x_3 &= \cos\theta_1 \left( L_2 \cos\theta_2 + L_3 \cos(\theta_2 + \theta_3) \right) \\
y_3 &= \sin\theta_1 \left( L_2 \cos\theta_2 + L_3 \cos(\theta_2 + \theta_3) \right) \\
z_3 &= L_1 + L_2 \sin\theta_2 + L_3 \sin(\theta_2 + \theta_3) \\
\end{aligned}
\]

\section{Ajuste los parámetros D-H según las medidas reales del robot que eligieron en laboratorio}

\section{Elabore un código Python que permita determinar las coordenadas p(x,y,z) del efector, en función de q1, q2 y q3}

\end{document}


